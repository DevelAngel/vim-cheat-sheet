\section{Keyboard macros}
\label{sec:keyboard-macros}

\begin{tabularx}{\textwidth}{l l X}
\toprule
MODE        & Shortcut & Description \tabularnewline
\midrule
\modenormal & \cmdsingle{\keyPoint*} & repeat last text manipulating command\tabularnewline
\addlinespace
\modenormal & \cmdsingle{q}[\keyChar*[a-z]] & start recording commands (in register \keyChar[a-z])\tabularnewline
\modenormal & \cmdsingle{q} & end recording\tabularnewline
\modenormal & \cmdsingle{@}[\keyChar*[a-z]] & play recorded commands (saved in register \keyChar[a-z])\tabularnewline
\bottomrule
\end{tabularx}

\medskip
Keyboard macro example (complex command sequence):
\begin{compactitem}
	\item Goal: \lstinline!stdio.h! \(\longrightarrow\) %
		    \lstinline!#include "stdio.h"!
	\item Keyboard macro:
		\begin{compactenum}
		\item \cmdsingle{q}[a] \cmdsep*
		\item \cmdsingle{\keyCircumflex*} \cmdsep*
		\item \cmdsingle{i}\lstinline!#include "!\keyEsc* \cmdsep*
		\item \cmdsingle{\$} \cmdsep*
		\item \cmdsingle{a}\lstinline!"!\keyEsc* \cmdsep*
		\item \cmdsingle{j} \cmdsep*
		\item \cmdsingle{q}
		\end{compactenum}
	\item Use keyboard macro with \cmdsingle{@}[a] or \cmdsingle*{@}[a]
\end{compactitem}


